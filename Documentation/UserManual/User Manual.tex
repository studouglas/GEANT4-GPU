\documentclass[12pt]{article}

% Packages
\usepackage[margin=1.2in]{geometry}
\usepackage{graphicx}
\usepackage{enumerate}
\usepackage{listings}
\usepackage{titling}
\usepackage{tabularx}
\usepackage{longtable}
\usepackage{booktabs}
\usepackage{hyperref}
\usepackage{makecell}
\usepackage{caption}
\usepackage{array}
\captionsetup[table]{skip=2pt}
% Comments --------------------------------------------------------------------
\usepackage{xcolor}
\newif\ifcomments\commentstrue
\ifcomments \newcommand{\authornote}[3]{\textcolor{#1}{[#3 ---#2]}}
\newcommand{\todo}[1]{\textcolor{red}{[TODO: #1]}} \else
\newcommand{\authornote}[3]{} \newcommand{\todo}[1]{} \fi
\newcommand{\wss}[1]{\authornote{magenta}{SS}{#1}}
\newcommand{\ds}[1]{\authornote{blue}{DS}{#1}}
\newcommand{\mmp}[1]{\authornote{green}{MP}{#1}}
% End Comments ---------------------------------------------------------------

\setlength\parindent{0pt} % Cleaner look


% Title Page -----------------------------------------------------------------
\title{
\LARGE GEANT-4 GPU Port:
\\\vspace{10mm}
\large \textbf{User Manual}
\vspace{40mm}
}
\author{
Stuart Douglas -- dougls2
\\Matthew Pagnan -- pagnanmm
\\Rob Gorrie -- gorrierw
\\Victor Reginato -- reginavp
\vspace{10mm}
}
\date{\vfill \textbf{Version 0}\\ \today}
% End Title Page -------------------------------------------------------------

% ============================== BEGIN DOCUMENT ============================= %
\begin{document}
\pagenumbering{gobble} % start numbering after TOC

% ============================== Title Page ============================= %
\maketitle
\newpage

% ================================= TOC ================================= %
\newgeometry{bottom=1.1in, top=1.1in}
\tableofcontents
\newpage
\pagenumbering{arabic}
\restoregeometry

% =============================== Section =============================== %
\section{Revision History}
All major edits to this document will be recorded in the table below.

\begin{table}[h]
\centering
\caption{Revision History}\label{Table_Revision}
\begin{tabular}{lll}
\toprule
\bf Description of Changes & \bf Author & \bf Date\\\midrule
Initial draft of document & Matt & 2016-02-26\\
\bottomrule
\end{tabular}
\end{table}

% =============================== Section =============================== %
\section{List of Figures}
\begin{center}
\begin{tabular}{cl}
\toprule
\bf Table \# & \bf Title\\\midrule
\bottomrule
\end{tabular}
\end{center}

\section{Definitions and Acronyms} % Matt
\begin{table}[h]
\centering
\caption{Definitions and Acronyms}
\begin{tabularx}{\textwidth}{l|X}
\Xhline{2\arrayrulewidth}
\bf Term & \bf Description\\
\hline
GEANT-4 & open-source software toolkit used to simulate the passage of particles through matter\\\hline
GEANT4-GPU & GEANT-4 with computations running on the GPU.\\\hline
G4-STORK & (Geant-4 STOchastic Reactor Kinetics), fork of GEANT-4 developed by McMaster's Engineering Physics department to simulate McMaster's nuclear reactor\\\hline
GPU & graphics processing unit, well-suited to parallel computing tasks\\\hline
CPU & computer processing unit, general computer processor well-suited to serial tasks\\\hline
CUDA & parallel computing architecture for general purpose programming on GPU, developed by NVIDIA\\\hline
FAQ & Frequently Asked Questions \\\hline
\Xhline{2\arrayrulewidth}
\end{tabularx}
\end{table}


% =============================== Section =============================== %

\section{Introduction} % Rob
\subsection{Purpose} % Rob
\subsection{Scope} % Rob	
\subsection{Background} % Rob
\subsection{Document Overview} % Matt
This Document goes over how to install and run GEANT4-GPU. As well as software and hardware required for GEANT4-GPU to be installed and run. Instructions on how to install the required software 
are also included in this document. Following the installation section, is a section on creating and running simulations using GEANT4-GPU.  There is a troubleshooting section as well as a FAQ 
in case you run into any problems installing or running GEANT4-GPU.

\section{Legal Information}	% Victor

\section{Installation} % Victor
\subsection{Required Hardware} % Victor
\subsection{Required Software} % Victor
\subsection{Supported Operating Systems} % Victor
\subsection{Installation Instructions} % Victor

\section{Execution} % Stuart
\subsection{Creating a Simulation} % Stuart
\subsection{Running the Simulation} % Stuart

\section{Porting Other Geant4 Modules to CUDA} % Stuart
\subsection{Layout of Source Files} % Stuart
\subsection{CMake Changes} % Stuart
\subsection{Interfacing with Existing Code} % Stuart
\subsection{Lessons Learned} % Stuart

\section{Troubleshooting} % Matt
\subsection{Installation} % Matt
\begin{itemize}
\item Ensure path names are correct, check for spaces in pathnames
\item The DCMAKE\_INSTALL\_PREFIX in step 4 of installing GEANT-4  has two paths. Make sure they are separated by a space.
\item make sure your operating system is supported 
\item Newer versions of Clang (included with Xcode 7) have been know to cause problems. Download Xcode 6 and uninstall Xcode 7 if this is the case.
\end{itemize}
\subsection{Running the Simulation} % Matt 
\begin{itemize}
\item Ensure that your graphics card is a NVidia card with Compute Capability 3.0.
\end{itemize}
\subsection{FAQ} % Matt
\textbf{What is GEANT-4?}\\
Many physics researchers use GEANT-4 to learn about how particles interact with a specific environment. It is a toolkit (i.e. library) that uses the Monte Carlo model, meaning each particle's properties are calculated independently according to certain probabilities. It runs all those calculations, and provides output\\

\textbf{Why will running the simulations on a GPU improve the performance?}\\
GPU's contain a large amount of cores that can perform calculations much more quickly than a CPU if the problem is well-suited to parallelization. GEANT-4 runs relatively simple calculations on millions of particles, and each particle is completely independent of the others. This is exactly that sort of well-suited problem, and stands to see large performance gains.\\

\textbf{Where can I find more information about GEANT-4?}\\
Cern has an entire website full of information for GEANT-4.\\ geant4.web.cern.ch/geant4/index.shtml\\

\textbf{Helpful Pages:}
\begin{itemize}
\item Download page for Geant4 source code as well as Data files\\ geant4.web.cern.ch/geant4/support/download.shtml
\item Getting Started\\ geant4.web.cern.ch/geant4/support/gettingstarted.shtml
\item Installation Guide for geant4 (does not include CUDA)\\ geant4.web.cern.ch/geant4/UserDocumentation/UsersGuides/InstallationGuide/html/index.html
\end{itemize}

\section{Appendix} % Rob
\subsection{Recommendations for Integration of Geant4 and CUDA}
\subsection{Future of Geant4-GPU}

\end{document}